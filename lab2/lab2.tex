\documentclass{ctexart}
\usepackage[utf8]{inputenc}
\usepackage{listings} 
\usepackage{xcolor}
\lstset{
  language=bash,  %代码语言使用的是matlab
  frame=shadowbox, %把代码用带有阴影的框圈起来
  rulesepcolor=\color{red!20!green!20!blue!20},%代码块边框为淡青色
  keywordstyle=\color{blue!90}\bfseries, %代码关键字的颜色为蓝色,粗体
  commentstyle=\color{red!10!green!70}\textit,    % 设置代码注释的颜色
  showstringspaces=false,%不显示代码字符串中间的空格标记
  numbers=left, % 显示行号
  numberstyle=\tiny,    % 行号字体
  stringstyle=\ttfamily, % 代码字符串的特殊格式
  breaklines=true, %对过长的代码自动换行
  extendedchars=false,  %解决代码跨页时,章节标题,页眉等汉字不显示的问题
%   escapebegin=\begin{CJK*},escapeend=\end{CJK*},      % 代码中出现中文必须加上,否则报错
  texcl=true}

\title{\vspace{-2cm}编写shell脚本}
\author{顾格非 \\ 3210103528}
\date{\today}

\begin{document}

\maketitle
\section{前言}
\subsection{动机}

\textbf{摘自MIT 6.null 导言: }
作为计算机科学家,我们都知道计算机最擅长帮助我们完成重复性的工作。 但是我们却常常忘记这一点也适用于我们使用计算机的方式,而不仅仅是利用计算机程序去帮我们求解问题。 在从事与计算机相关的工作时,我们有很多触手可及的工具可以帮助我们更高效的解决问题。 但是我们中的大多数人实际上只利用了这些工具中的很少一部分,我们常常只是死记硬背一些如咒语般的命令, 或是当我们卡住的时候,盲目地从网上复制粘贴一些命令。
\subsection{结缘}
在准备六级考试的时候,为了提高我的听力,我选择了MIT-Missing-Semester,这门课旨在介绍一些上课不会教的工具,比如vim,Docker,Git…而shell编程就是其中之一。

在王何宇老师的《数学软件》,我再一次接触了shell。十分感谢这门课程,下面是我的工作。
\subsection{我的工作}
之前接触了Python和C等高级语言,这里我尝试去理解和运用shell语言。
\begin{enumerate}
    \item 我学习了shell语法,简单掌握了循环,判断等结构,下面我会在\textbf{用shell语法实现冒泡排序}。
    \item shell语法也有让一般编程费解的地方,\textbf{我复现了书上从函数中返回值的实验},并展示了结果。
\end{enumerate}

\section{正文}
\subsection{冒泡排序}
冒泡排序的思想即每次排好一个数,限于篇幅我也不多介绍了吧。
\subsection{脚本代码}


\begin{verbatim}
#!/bin/bash
arr=(78 21 34 11 213 44)
echo "你的数组是:${arr[*]}"
for ((i=0;i<${#arr[*]};i++))
do
  for ((a=$i+1;a<${#arr[*]};a++))
  do
#如果前一个元素大于后一个元素,就交换位置,否则下一次循环
  if [ ${arr[$i]} -gt ${arr[$a]} ];then
    temp=$arr[$i]}
    arr[$i]=${arr[$a]}
    arr[$a]=$temp
  fi
  done
done
echo "你的数组升序排列为:${arr[*]}"
\end{verbatim}

\subsection{结果展示}
将脚本命名为\verb|1.sh|,采用\verb|bash 1.sh|编译,结果如下:
\begin{verbatim}
    你的数组是:78 21 34 11 213 44
    你的数组升序排列为:11 21 34 44 78 213
\end{verbatim}
\newpage
\vspace{-0.2cm}
\section{正文2}
\subsection{简单介绍}
这是书上从函数中返回一个值的实验。在shell头文件之后,我们定义了一个函数\verb|yes_or_no()|,在脚本开始执行时,函数被定义而不立即执行,而后面传入参数给这个函数,最终得到想要的结果。
\subsection{脚本代码}
这里换了一种好看的代码块!
\begin{lstlisting}
#!/bin/bash
yes_or_no() {
  echo "Is your name $* ?"
  while true
  do
    echo -n "Enter yes or no: "
    read x
    case "$x" in
      y | yes ) return 0;;
      n | no ) return 1;;
      * )    echo "Answer yes or no"
    esac
  done
}

echo "Original parameters are $*"
if yes_or_no "$1"
then 
  echo "Hi $1, nice name"
else
  echo "Never mind"
fi
exit 0
\end{lstlisting}

\subsection{结果展示}
这里我在终端中输入\verb|bash 2.sh frank|,(其中frank是像函数传入的参数作为名字)
\begin{verbatim}
Original parameters are frank
Is your name frank ?
Enter yes or no: yes
Hi frank, nice name
\end{verbatim}
\end{document}
