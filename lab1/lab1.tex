\documentclass{ctexart}

\usepackage{graphicx}
\usepackage{amsmath}


\title{\vspace{-2cm}实对称矩阵必可在实数域上相似对角化}


\author{顾格非 \\ 统计学 3210103528}

\begin{document}

\maketitle


这是一个来自数分的问题,也是我《数学软件》的作业1.
\section{问题描述}
实对称矩阵在实数域上一定能相似对角化吗?

答案是肯定的。下面我们就来证明它吧!

\section{证明}
我们只要证明存在n阶的正交阵$U$使得$U^TAU$为对角阵即可。为此,我们对矩阵的阶作归纳。

若$A$是1阶方阵,它已经对角化,令$U=(1)_{1x1}$即得。

设已证明任何一个$n-1$阶实对称阵都存在相应的正交阵$U_1$使得$U_1^TAU$为对角阵.首先对于任何一个$n$阶实对称阵$A$,它有$n$个实特征值(包括重数)。设$\lambda_1$为其中一个特征值,$\xi$为$A$的属于$\lambda_1$的一个特征向量且$|\xi_1 | = 1$,用Schmidt正交化方法将$\xi_1$扩充为$R^n$中的一组标准正交集$\xi_1,\xi_2,…,\xi_n$,则$A\xi_1,A\xi_2,…,A\xi_n$均可以用$\xi_1,\xi_2,…,\xi_n$线性表示,不难验证:
\begin{equation}
    A(\xi_1,\xi_2,…,\xi_n)=(A\xi_1,A\xi_2,…,A\xi_n)=(\xi_1,\xi_2,…,\xi_n)\begin{pmatrix}
  \lambda_1 & a\\
  0  & A_1
\end{pmatrix}
\end{equation}

令$U_0=(\xi_1,\xi_2,…,\xi_n)$,则因为$\xi_1,\xi_2,…,\xi_n$为标准正交基知$U_0$为正交阵,故(1)等价与
\begin{equation}
    U_0^TAU_0=\begin{pmatrix}
  \lambda_1 & a\\
  0  & A_1
\end{pmatrix}
\end{equation}
由于(2)等式左端为实对称阵,故其等式右端的矩阵也是实对称的,从而$a$为$n-1$维的零向量,$A_1$为$n-1$阶的实对称阵,依归纳假设知,存在$n-1$阶正交阵$U_1$及$n-1$阶对角阵$\Lambda_1$,使得$U_1^TAU_1=\Lambda_1$,令
\begin{equation}
    U=U_0\begin{pmatrix}
  1 & 0\\
  0  & 1
\end{pmatrix}
\end{equation}
则$U$为正交阵,且
\begin{equation}
    U^TAU=\begin{pmatrix}
  1 & 0\\
  0  & U_1^T
\end{pmatrix}
 \begin{pmatrix}
  \lambda_1 & 0\\
  0  & A_1
\end{pmatrix}
 \begin{pmatrix}
 1 & 0\\
  0  & U_1
\end{pmatrix}=\begin{pmatrix}
 \lambda_1 & 0\\
  0  & \Lambda_1
\end{pmatrix}
\end{equation}
这说明所要证明的结论对$n$阶实对称阵$A$也成立。由数学归纳法,对所有$n$阶的实对称阵$A$,均存在一个n阶的正交阵$U$使得$U^TAU$为对角阵,证毕。
\end{document}